\documentclass[12pt, openany]{book} 
\usepackage[utf8]{inputenc}
\usepackage[letterpaper, margin=1in]{geometry}

\usepackage[spanish,english]{babel}

\usepackage{graphicx}
\usepackage{titlesec}
\usepackage{tocloft}%
\usepackage{algorithm} 
\usepackage{algpseudocode}
\usepackage[backend=biber, style=apa, natbib=true]{biblatex}
\usepackage{setspace}

\usepackage{amsmath}
\usepackage{listings}
\usepackage{color}
\usepackage{subcaption}


\usepackage{amsmath}
\usepackage{amsfonts}
\usepackage{array}
\usepackage{booktabs}
\usepackage{graphicx}
\usepackage{caption}

\definecolor{codegreen}{rgb}{0,0.6,0}
\definecolor{codegray}{rgb}{0.5,0.5,0.5}
\definecolor{codepurple}{rgb}{0.58,0,0.82}
\definecolor{backcolour}{rgb}{0.95,0.95,0.92}
%\definecolor{backcolour}{rgb}{0.15, 0.15, 0.15}
%\definecolor{codewhite}{rgb}{0.9, 0.9, 0.9}

\lstdefinestyle{mystyle}{
    backgroundcolor=\color{backcolour},   
    commentstyle=\color{codegreen},
    keywordstyle=\color{magenta},
    numberstyle=\tiny\color{codegray},
    stringstyle=\color{codepurple},
    basicstyle=\ttfamily\footnotesize,
    breakatwhitespace=false,         
    breaklines=true,                 
    captionpos=b,                    
    keepspaces=true,                 
    numbers=left,                    
    numbersep=5pt,                  
    showspaces=false,                
    showstringspaces=false,
    showtabs=false,                  
    tabsize=2
}
\lstset{style=mystyle}


\addbibresource{referencias.bib} 
\defbibheading{bibnumbered}{\section{Fuentes de Información}}

\graphicspath{{/Users/mijatola/Documents/Tesis/}} % Asegúrate de que la ruta termine en /
\selectlanguage{spanish}
\let \savenumberline \numberline
\def \numberline#1{\savenumberline{#1.}}

\pagenumbering{roman}

\title{
    \textbf{
      \Large{UNIVERSIDAD MAYOR DE SAN ANDR\'ES}\\
      \normalsize{
        FACULTAD DE CIENCIAS PURAS Y NATURALES\\
        CARRERA DE INFORM\'ATICA\\
      }
      \hfill \break
      \includegraphics[width=5cm,height=11cm]{umsa}
      \break
      \begin{large}
        EVALUACION COMPARATIVA DE MODELOS GENERADORES DE GRAFOS ALEATORIOS PARA REDES COMPLEJAS
      \end{large}
      \break
      \small{
        Tesis de Grado para obtener el Título de Licenciatura en Informática \break
        Mención Ingeniería de Sistemas Informáticos\break
      }
      \break
      \large {
        POR: CARLOS MIJAEL TOLA APAZA\\
        TUTOR: M.Sc. JORGE TERAN
      }
      \break
      \small {
        LA PAZ - BOLIVIA \break
        Abril, 2024
      }
    }
}

\author{}
\date{}

\begin{document}

\maketitle

\selectlanguage{spanish}

\newpage
\thispagestyle{empty}
\vspace*{\fill} 

\begin{flushright}
  \textbf{Dedicatoria} \\
\textit{Dedicado a mi familia, por creer en mí;\\
 a mis profesores, por iluminar el camino;\\ 
a mi esposa e hijo, \\ 
por estar conmigo en cada paso de  \\
este viaje.}
\end{flushright}

\vspace{2cm} 
\newpage

\thispagestyle{empty} 

\vspace*{\fill} 

\begin{flushleft}
\textbf{Agradecimientos}\\ 
\textit{Quiero expresar mi más sincero agradecimiento a...}\\
A mis padres por apoyarme a lo largo de mi carrera. \\
A mi asesor M. Sc Jorge Teran Pomier por su apoyo y confianza en el proceso de elaboracion
de la presente tesis.\\
Al lector, gracias por darle vida y sentido a este trabajo.
\end{flushleft}

\vspace*{\fill} % 

\newpage

\tableofcontents



\doublespacing
\pagenumbering{arabic}

\chapter{Marco Referencial}
\section{Introducción}
La teoría de grafos es una herramienta poderosa en diversas áreas de la ciencia y la ingeniería, es una rama de las matemáticas y las ciencias de la 
computación que se ocupa del estudio de grafos, que son estructuras compuestas por nodos conectados por aristas. Este campo encuentra aplicaciones en 
diversas áreas, incluyendo la optimización de redes, la sociología, la planificación urbana, la biología computacional, y más. Los grafos aleatorios, 
en particular, son un tipo de grafo donde la estructura tiene algún elemento de aleatoriedad, lo que los hace útiles para modelar situaciones reales en las que el diseño exacto del grafo no se conoce de antemano \citep{Newman2010}.
En las últimas décadas, los modelos generadores de grafos aleatorios han adquirido una gran relevancia en el análisis de redes complejas, permitiendo
 a los investigadores comprender mejor las propiedades emergentes de estos sistemas. Las redes complejas abarcan una amplia gama de sistemas, desde redes sociales y de comunicación hasta redes biológicas y tecnológicas. Un aspecto crucial del estudio de estas redes es identificar y comprender las propiedades estructurales que emergen de la interacción entre sus componentes.

El modelo de Erdős-Rényi, introducido en la década de 1960, fue uno de los primeros intentos de formalizar la generación de grafos aleatorios. 
Este modelo se basa en la premisa de que cada par de nodos tiene una probabilidad fija de estar conectado por una arista, lo que conduce a una distribución binomial de los grados de los nodos \citep{Erdos1959}. Aunque este modelo es matemáticamente simple y proporciona una base sólida para el estudio de grafos aleatorios, tiene limitaciones en su capacidad para replicar algunas de las propiedades observadas en redes reales, como la distribución de grados de ley de potencia y el alto coeficiente de agrupamiento.

Para abordar algunas de estas limitaciones, surgieron modelos más sofisticados, como el modelo de Barabási-Albert y el modelo de Watts-Strogatz. El modelo de Barabási-Albert introduce los conceptos de crecimiento y preferencia de conexión, resultando en una distribución de grados que sigue una ley de potencia \citep{Barabasi1999}. Este modelo ha sido fundamental para explicar la presencia de ''hubs'' o nodos altamente conectados en redes reales, una característica que no se observa en el modelo de Erdős-Rényi.

Por otro lado, el modelo de Watts-Strogatz busca capturar la propiedad de ''pequeño mundo'' observada en muchas redes reales, donde los nodos están altamente conectados localmente pero también presentan caminos cortos entre pares de nodos distantes \citep{Watts1998}. Este modelo combina la regularidad estructural con elementos aleatorios, proporcionando una mejor representación del agrupamiento y las distancias cortas típicas de las redes sociales y biológicas.

El estudio de estos modelos generadores de grafos aleatorios no solo es relevante desde un punto de vista teórico, sino que también tiene importantes implicaciones prácticas. En el ámbito de las redes de comunicación, por ejemplo, comprender las propiedades estructurales puede ayudar a diseñar redes más robustas y eficientes. En biología, los modelos de grafos aleatorios pueden ser utilizados para entender la organización de redes de interacción de proteínas o redes neuronales, facilitando el desarrollo de nuevas terapias y tratamientos.

A pesar de los avances significativos en el modelado de grafos aleatorios, sigue existiendo una brecha en la comprensión de cuál de estos modelos es más adecuado para diferentes tipos de redes y aplicaciones. La selección del modelo correcto puede tener un impacto significativo en la precisión de las simulaciones y en la capacidad para predecir comportamientos emergentes en redes complejas. Esta tesis se centra en la evaluación comparativa de los modelos de Erdős-Rényi, Barabási-Albert y Watts-Strogatz, utilizando un conjunto de métricas clave que incluyen el coeficiente de agrupamiento, la longitud promedio del camino más corto, la distribución de grados y la robustez frente a fallos y ataques.

El objetivo principal de este estudio es identificar las fortalezas y debilidades de cada modelo, proporcionando una guía para seleccionar el modelo más adecuado en función de las características específicas de la red bajo estudio. A través de simulaciones computacionales y análisis estadísticos, se busca ofrecer una comprensión más profunda de cómo estos modelos pueden ser aplicados de manera efectiva en diferentes contextos.

En los siguientes capítulos, se explorarán en detalle los fundamentos teóricos de cada modelo, la metodología utilizada para su implementación y los resultados de los experimentos comparativos. Se espera que los hallazgos de esta investigación contribuyan al desarrollo de mejores herramientas y técnicas para el análisis de redes complejas, así como a una mayor comprensión de las dinámicas subyacentes en estos sistemas.

\section{Problema}
\subsection{Antecedentes}
Los modelos de grafos aleatorios han sido un área de interés en la teoría de grafos y las ciencias de la 
computación durante décadas, ofreciendo conocimiento valiosos en el estudio de redes complejas. El modelo
 de Erdős-Rényi, introducido en la década de 1960, es uno de los primeros y más estudiados modelos de grafos aleatorios, marcando un punto de partida para la investigación en este campo \citep{Erdos1959}.

Las siguientes investigaciones proporcionan un marco importante para el estudio de modelos generadores de grafos aleatorios y han influido significativamente en el desarrollo de este campo:

\begin{itemize}
    \item \textbf{Título:} ``On Random Graphs I'' \\
    \textbf{Autor:} P. Erdős y A. Rényi \\
    \textbf{Año:} 1959 \\
    \textbf{Institución:} Universidad de Debrecen \\
    \textbf{Resumen:} Este trabajo introduce el modelo $G(n, p)$, explorando las propiedades emergentes de los grafos aleatorios a medida que la probabilidad de conexión entre pares de nodos varía. Es fundamental para comprender la fase de transición en la aparición de componentes gigantes dentro de redes complejas \citep{Erdos1959}.

    \item \textbf{Título:} ``Emergence of Scaling in Random Networks'' \\
    \textbf{Autor:} A.-L. Barabási y R. Albert \\
    \textbf{Año:} 1999 \\
    \textbf{Institución:} Universidad de Notre Dame \\
    \textbf{Resumen:} Este artículo presenta el modelo de conexión preferencial, ilustrando cómo las redes reales tienden a formar una estructura de red libre de escala. El estudio demuestra que la dinámica de crecimiento y la conexión preferencial son mecanismos clave en la formación de redes con distribuciones de grado de ley de potencia \citep{Barabasi1999}.

    \item \textbf{Título:} ``Collective dynamics of ‘small-world’ networks'' \\
    \textbf{Autor:} D.J. Watts y S.H. Strogatz \\
    \textbf{Año:} 1998 \\
    \textbf{Institución:} Universidad de Cornell \\
    \textbf{Resumen:} Este documento introduce el modelo de mundo pequeño, destacando cómo puede lograrse un alto coeficiente de agrupamiento y caminos cortos promedio simultáneamente en redes. Los autores muestran que esta estructura es prevalente en muchas redes del mundo real, desde la red eléctrica hasta las redes neuronales \citep{Watts1998}.
\end{itemize}

Cada una de estas investigaciones contribuye a la base teórica y metodológica para el análisis y la comparación de modelos generadores de grafos aleatorios, destacando la diversidad y la evolución de los enfoques en este campo de estudio.


\subsection{Planteamiento del Problema}
A pesar de la existencia de varios modelos generadores de grafos aleatorios, hay una falta de comprensión integral sobre cuál de estos modelos ofrece la mejor representación y eficiencia en la simulación de redes complejas en diversos contextos. La selección adecuada de un modelo generador de grafos es crucial para investigadores y profesionales que buscan analizar y predecir comportamientos en redes complejas, tales como redes sociales, biológicas, o tecnológicas. Sin embargo, la variedad de modelos disponibles y sus distintas propiedades y aplicaciones hacen difícil determinar cuál es más adecuado para un propósito específico. Esta problemática se complica aún más por la ausencia de una comparación exhaustiva y sistemática que evalúe estos modelos bajo un conjunto común de criterios y en situaciones aplicables a la vida real \citep{Wilson2008, Bollobas2001, Caldarelli2007} .
%A pesar de los avances significativos en el modelado de grafos aleatorios y su aplicación en diversos campos, persisten desafíos en la selección del modelo más adecuado para representar estructuras de red específicas.

\subsection{Formulacion del problema}
\textquestiondown Qu\'e modelo generador de grafos aleatorios de los estudiados es el mejor para diseñar redes complejas?


\section{Objetivos}
\subsection{Objetivo General}
Comparar distintos modelos generadores de grafos aleatorios para identificar sus fortalezas, debilidades y aplicaciones óptimas .
\subsection{Objetivos Específicos}
\begin{itemize}
    \item Describir los modelos generadores de grafos aleatorios más utilizados .
    \item Implementar una serie de experimentos para comparar los modelos en términos de características de grafos .
    \item Analizar la aplicabilidad de cada modelo en contextos específicos basados en los resultados obtenidos .
\end{itemize}

\section{Hipótesis}
Entre los modelos generadores de grafos aleatorios estudiados, existe al menos uno que,
bajo un conjunto definido de criterios de evaluación como la precisión en la representación de las propiedades 
estructurales de las redes complejas, eficiencia computacional y aplicabilidad en diversos contextos, demuestra ser 
el mejor para simular redes complejas.

\section{Justificación}
\subsection{Justificación Económica}
El desarrollo de modelos más precisos puede tener un impacto económico considerable en industrias que dependen de la optimización de redes.
\subsection{Justificación Social}
Esta investigación contribuye al entendimiento de las redes sociales y su dinámica, lo que puede tener implicaciones en la formulación de políticas públicas y la gestión de crisis.
\subsection{Justificación Científica}
Científicamente, este estudio contribuye al cuerpo de conocimiento en teoría de grafos y ciencias de la computación .

\section{Alcances}
El estudio se enfoca en los modelos generadores de grafos aleatorios más reconocidos y aplicados en el campo.\\
La investigación se limita a la comparación de modelos basados en los criterios predefinidos y no explorará variantes menos conocidas .

\section{Metodología}

Esta investigación adopta un enfoque cuantitativo para comparar los modelos generadores de grafos aleatorios más utilizados, utilizando tanto análisis teóricos como simulaciones computacionales. La metodología se divide en varias fases clave para asegurar una evaluación exhaustiva y sistemática de cada modelo .

\subsection{Selección de Modelos}

Los modelos seleccionados para esta comparación incluyen el modelo de Erdős-Rényi, el modelo de Barabási-Albert y el modelo de Watts-Strogatz, debido a su prevalencia en la literatura y su relevancia en la modelización de redes complejas . Estos modelos se escogen por su capacidad para representar diferentes aspectos de las redes, como la aleatoriedad, el crecimiento y la formación de agrupamientos .

\subsection{Criterios de Evaluación}

Para comparar los modelos, se establecen varios criterios de evaluación basados en las propiedades de los grafos generados, incluyendo:
\begin{itemize}
    %\item Distribución de grados
    \item Coeficiente de agrupamiento
    \item Longitud promedio de caminos
    %\item Robustez de la red frente a fallos y ataques
\end{itemize}

\subsection{Recopilación y Análisis de Datos}

El análisis de los modelos se basará en:
\begin{enumerate}
    \item Simulaciones computacionales: Utilizando software especializado en teoría de grafos, se generarán grafos aleatorios basados en cada modelo, con diferentes parámetros, para analizar las propiedades mencionadas.
    \item Comparación con datos reales: Se utilizarán conjuntos de datos de redes reales disponibles públicamente para comparar las características de los grafos generados por los modelos con las propiedades de redes complejas reales.
\end{enumerate}

\subsection{Herramientas de Software}

Se emplearán herramientas y bibliotecas de software como Python para la generación de grafos y la libreria Sympy para el cálculo de métricas estadisticas facilitando la simulación y análisis de los modelos .

\subsection{Evaluación y Comparación de Modelos}

Los resultados obtenidos de las simulaciones y el análisis de datos se compararán utilizando métodos estadísticos para determinar cuál de los modelos proporciona una representación más precisa de las redes complejas según los criterios establecidos .

\subsection{Limitaciones Metodológicas}

Se reconocen limitaciones en la metodología, incluyendo la dependencia de la calidad de los datos de redes reales y las simplificaciones inherentes a los modelos teóricos. Se tomarán medidas para mitigar estas limitaciones, como la verificación de resultados mediante múltiples simulaciones y la comparación con estudios previos .

Esta metodología está diseñada para proporcionar una comprensión clara y objetiva de cómo diferentes modelos generadores de grafos aleatorios replican las propiedades de las redes complejas, con el fin de identificar las herramientas más adecuadas para aplicaciones específicas en la investigación y la industria .

\chapter{Marco Teórico}

\section{Introducción a la Teoría de Grafos}
\subsection{Definición y Conceptos Básicos}

La teoría de grafos es una rama de las matemáticas y las ciencias de la computación que estudia las propiedades de los grafos. Un \textbf{grafo} $G$ se define formalmente como un par ordenado $G = (V, E)$ compuesto por un conjunto $V$ de vértices o nodos y un conjunto $E$ de aristas o enlaces, donde una arista es un par de vértices que representa la conexión entre ellos. Los grafos se clasifican en varias categorías dependiendo de sus características específicas:

\begin{itemize}
    \item \textbf{Grafos dirigidos y no dirigidos:} En un grafo dirigido, las aristas tienen una dirección, indicando la relación de un vértice a otro. En contraste, las aristas de un grafo no dirigido no tienen dirección .
    \item \textbf{Grafos ponderados y no ponderados:} Un grafo ponderado asigna un peso o costo a cada arista, que puede representar, por ejemplo, la distancia entre dos puntos. Los grafos no ponderados no asignan estos pesos .
    \item \textbf{Grafos simples:} Un grafo simple no permite bucles (aristas que conectan un vértice consigo mismo) ni múltiples aristas entre el mismo par de vértices .
\end{itemize}

Además, algunos conceptos clave en la teoría de grafos incluyen:

\begin{itemize}
    \item \textbf{Caminos:} Una secuencia de vértices donde cada par consecutivo de vértices está conectado por una arista .
    \item \textbf{Ciclos:} Un camino que comienza y termina en el mismo vértice, sin repetir ningún vértice o arista .
    \item \textbf{Grafos conexos:} Un grafo no dirigido es conexo si existe un camino entre cualquier par de vértices. En grafos dirigidos, esta propiedad se denomina fuertemente conexo .
    \item \textbf{Subgrafos:} Un subgrafo es un grafo cuyos vértices y aristas son todos subconjuntos de otro grafo .
    \item \textbf{Grafos complementarios:} Dado un grafo simple $G = (V, E)$, su grafo complementario es un grafo que tiene los mismos vértices que $G$ pero cuyas aristas son aquellas que no están presentes en $G$ .
\end{itemize}

La teoría de grafos se aplica en numerosas disciplinas para modelar relaciones entre entidades en redes de transporte, comunicaciones, biología, informática, y más, ofreciendo herramientas poderosas para resolver problemas complejos en ciencia e ingeniería \citep{West2000, Gross2004, Castillo2005} .

\section{Historia y Desarrollo}

La teoría de grafos, como campo matemático formal, tuvo sus inicios en el siglo XVIII con el trabajo del matemático suizo Leonhard Euler. En 1736, Euler abordó el famoso \textit{Problema de los Puentes de Königsberg}, considerado el primer teorema de la teoría de grafos. El problema preguntaba si era posible cruzar los siete puentes de la ciudad de Königsberg sin cruzar ninguno más de una vez y regresar al punto de partida. Euler demostró que esto no era posible, introduciendo el concepto de grafos en el proceso \citep{Wilson2008}.
Este trabajo pionero sentó las bases para el desarrollo de la teoría de grafos. Durante el siglo XIX y principios del XX, el campo creció lentamente, pero la introducción de la computación y la necesidad de algoritmos eficientes en la segunda mitad del siglo XX aceleraron su desarrollo. La teoría de grafos se convirtió en una herramienta crucial para la informática, especialmente en áreas como la teoría de algoritmos, estructuras de datos, y la optimización de redes .
Un hito importante en el desarrollo de la teoría de grafos fue el \textit{Teorema de los Cuatro Colores}, propuesto por primera vez en 1852, que afirma que cualquier mapa plano puede ser coloreado con no más de cuatro colores de manera que no haya dos regiones adyacentes del mismo color. Aunque el teorema fue propuesto en el siglo XIX, no se demostró completamente hasta 1976, utilizando la ayuda de computadoras por Kenneth Appel y Wolfgang Haken \citep{Wilson2008} .
En las décadas siguientes, la teoría de grafos se ha aplicado a una gama cada vez mayor de problemas en ciencias de la computación, biología, ingeniería de redes, sociología y muchas otras disciplinas. Los modelos generadores de grafos aleatorios, introducidos por Paul Erdős y Alfréd Rényi en los años 1950 y 1960, han permitido a los investigadores estudiar propiedades estadísticas de las redes complejas. Más recientemente, el descubrimiento de las propiedades de los "pequeños mundos" por Duncan Watts y Steven Strogatz, y el estudio de las "redes sin escala" por Albert-László Barabási y Réka Albert, han revolucionado nuestra comprensión de las redes complejas en el mundo real, desde Internet hasta las redes sociales y las redes biológicas \citep{Newman2010, Bollobas2001, Caldarelli2007} .
Hoy en día, la teoría de grafos sigue siendo un área de investigación vibrante y en expansión, impulsando avances en la matemática pura y aplicada, y ofreciendo nuevas herramientas para abordar problemas complejos en una variedad de campos científicos y de ingeniería \citep{West2000, Gross2004, Castillo2005} .

\section{Definiciones Importantes en Teoría de Grafos}

\subsection{Camino Corto}

Un \textbf{camino corto} (shortest path) entre dos nodos en un grafo es el camino que tiene el menor número de aristas. Formalmente, en un grafo \( G = (V, E) \), el camino corto \( P_{uv} \) entre dos nodos \( u \) y \( v \) es el camino tal que la suma de las longitudes de sus aristas es mínima:

\[
d(u,v) = \min \{ \sum_{i=1}^{k} w(e_i) \mid e_i \in E \}
\]

donde \( w(e_i) \) es el peso de la arista \( e_i \) y \( d(u,v) \) es la distancia entre los nodos \( u \) y \( v \).

\subsection{Búsqueda en Anchura (BFS)}

La \textbf{búsqueda en anchura} (Breadth-First Search, BFS) es un algoritmo para recorrer o buscar en un grafo. Comienza en el nodo raíz y explora todos los nodos vecinos en cada nivel antes de pasar al siguiente nivel. El algoritmo BFS se puede definir recursivamente como:

\begin{algorithm}
\caption{BFS}
\begin{algorithmic}[1]
\Procedure{BFS}{$G, s$}
    \State \textbf{Input:} Un grafo \( G = (V, E) \), un nodo inicial \( s \)
    \State \textbf{Output:} Distancias mínimas desde \( s \) a todos los otros nodos
    \State \textbf{Inicializar} todas las distancias \( d(v) \) a \(\infty\) y \( d(s) \) a 0
    \State \textbf{Crear una cola} \( Q \) y \textbf{encolar} \( s \)
    \While{Q no esté vacía}
        \State \( u \gets \text{desencolar} \ Q \)
        \For{cada vecino \( v \) de \( u \)}
            \If{\( d(v) = \infty \)}
                \State \( d(v) \gets d(u) + 1 \)
                \State \text{encolar} \( v \)
            \EndIf
        \EndFor
    \EndWhile
\EndProcedure
\end{algorithmic}
\end{algorithm}

\subsection{Ciclos}

Un \textbf{ciclo} en un grafo es una secuencia de aristas que comienza y termina en el mismo nodo, sin repetir ninguna arista. Formalmente, un ciclo en un grafo \( G \) es una sucesión de nodos \( v_1, v_2, \ldots, v_k \) tal que:

\[
v_1 = v_k \text{ y } \forall i \in \{1, \ldots, k-1\}, (v_i, v_{i+1}) \in E \text{ y } v_i \neq v_j \text{ para } i \neq j
\]

\subsection{Grafo Completo}

Un \textbf{grafo completo} es un grafo en el que cada par de nodos distintos está conectado por una única arista. Formalmente, un grafo completo \( K_n \) con \( n \) nodos es un grafo tal que:

\[
E = \{ (u, v) \mid u, v \in V \text{ y } u \neq v \}
\]

El número total de aristas en un grafo completo es:

\[
|E| = \frac{n(n-1)}{2}
\]

\subsection{Subgrafo}

Un \textbf{subgrafo} es un grafo cuyos nodos y aristas son subconjuntos de los nodos y aristas de otro grafo. Formalmente, un grafo \( H = (W, F) \) es un subgrafo de \( G = (V, E) \) si:

\[
W \subseteq V \text{ y } F \subseteq E
\]

\subsection{Distancia}

La \textbf{distancia} entre dos nodos \( u \) y \( v \) en un grafo es la longitud del camino más corto entre ellos, denotada como \( d(u,v) \). Para un grafo no ponderado, la distancia es el número de aristas en el camino más corto:

\[
d(u,v) = \min \{ \text{longitud del camino} \ P_{uv} \mid P_{uv} \text{ es un camino entre } u \text{ y } v \}
\]

En grafos ponderados, la distancia es la suma de los pesos de las aristas en el camino más corto.


\section{Redes de Mundo Pequeño}

Las redes de mundo pequeño son un tipo de estructura de red caracterizada por un alto coeficiente de agrupamiento y una corta longitud promedio del camino más corto. Este concepto fue introducido por Duncan Watts y Steven Strogatz en 1998 \citep{watts1998collective}. Las redes de mundo pequeño combinan propiedades de grafos regulares y aleatorios, y se observan en muchas redes reales, incluyendo redes sociales, redes biológicas y redes de transporte.



\section{Modelos Generadores de Grafos Aleatorios}

\subsection{Modelo de Erdős-Rényi}

El Modelo de Erdős-Rényi, nombrado así por los matemáticos Paul Erdős y Alfréd Rényi, es uno de los primeros modelos propuestos para la generación de grafos aleatorios y sigue siendo uno de los más estudiados en la teoría de grafos. Este modelo se presenta en dos variantes: $G(n, p)$ y $G(n, M)$ .

En la variante $G(n, p)$, un grafo es generado comenzando con un conjunto de $n$ vértices y conectando cada par de vértices distintos con probabilidad $p$ independientemente de los demás pares. En la variante $G(n, M)$, un grafo es generado comenzando con un conjunto de $n$ vértices y añadiendo exactamente $M$ aristas entre pares de vértices seleccionados al azar sin repetición \citep{Erdos1959} .

\paragraph{Pseudocódigo para el Modelo $G(n, p)$}

A continuación, se proporciona el pseudocódigo para generar un grafo basado en el modelo $G(n, p)$:

\begin{algorithm}
\caption{Generación de Grafo Aleatorio según el Modelo de Erdős-Rényi $G(n, p)$}
\begin{algorithmic}[1]
\State \textbf{Entrada:} Número de vértices $n$, probabilidad $p$
\State \textbf{Salida:} Grafo $G$ generado según $G(n, p)$
\Procedure{GenerarGrafoErdosRenyi}{$n, p$}
    \State Inicializar grafo $G$ con $n$ vértices y sin aristas
    \For{cada par de vértices $i, j$ en $G$, con $i < j$}
        \State Generar un número aleatorio $r$ en el intervalo $[0, 1]$
        \If{$r < p$}
            \State Añadir arista $(i, j)$ al grafo $G$
        \EndIf
    \EndFor
    \State \textbf{return} $G$
\EndProcedure
\end{algorithmic}
\end{algorithm}


\subsection{Modelo de Barabási-Albert (Conexión Preferencial)}

El Modelo de Barabási-Albert es conocido por introducir los conceptos de crecimiento y conexión preferencial en la generación de grafos, lo que lleva a la formación de redes libres de escala. Estas redes se caracterizan por una distribución de grado que sigue una ley de potencias, típica de muchas redes en el mundo real, como internet, redes sociales y redes biológicas \citep{Barabasi1999} .

\paragraph{Pseudocódigo para el Modelo de Barabási-Albert}

El siguiente pseudocódigo describe el proceso de generación de un grafo según el Modelo de Barabási-Albert:

\begin{algorithm}
\caption{Generación de Grafo según el Modelo de Barabási-Albert}
\begin{algorithmic}[1]
\State \textbf{Entrada:} Número inicial de vértices $m_0$, número de aristas a añadir por cada nuevo vértice $m (m \leq m_0)$
\State \textbf{Salida:} Grafo $G$ generado según Barabási-Albert
\Procedure{GenerarGrafoBarabasiAlbert}{$m_0, m$}
    \State Inicializar grafo $G$ con $m_0$ vértices conectados de manera arbitraria
    \For{cada nuevo vértice $v$ a añadir al grafo}
        \For{cada vértice $u$ ya existente en $G$}
            \State Calcular la probabilidad $p_u$ de conectar el nuevo vértice $v$ con $u$ basada en el grado de $u$
            \State Generar un número aleatorio $r$ en el intervalo $[0, 1]$
            \If{$r < p_u$}
                \State Añadir arista entre $v$ y $u$
            \EndIf
        \EndFor
    \EndFor
    \State \textbf{return} $G$
\EndProcedure
\end{algorithmic}
\end{algorithm}

La probabilidad $p_u$ de conectar el nuevo vértice $v$ con un vértice existente $u$ es proporcional al grado de $u$, lo que refleja el mecanismo de "conexión preferencial" .

\paragraph{Complejidad Algorítmica}

La complejidad algorítmica del modelo de Barabási-Albert depende principalmente de dos factores: el número inicial de vértices $m_0$ y el número de aristas $m$ que se añaden por cada nuevo vértice. Para cada nuevo vértice, el algoritmo debe calcular la probabilidad de conexión para cada vértice existente, lo que implica una operación por cada uno de los vértices existentes en el grafo hasta ese momento. Si $N$ es el número total de vértices en el grafo final, la complejidad algorítmica del proceso de generación del grafo es $O(N \cdot m_0)$ para los pasos iniciales, pero considerando la conexión preferencial y el crecimiento del grafo, puede aproximarse a $O(N^2)$ en escenarios donde $m$ y $m_0$ son comparativamente pequeños frente a $N$. Sin embargo, en la práctica, el proceso es más eficiente que $O(N^2)$ debido a que $m$ suele ser mucho menor que $N$ .

Este modelo es especialmente interesante por su capacidad para generar redes que imitan la estructura de muchas redes complejas observadas en la naturaleza y la sociedad, destacando la importancia de los mecanismos de crecimiento y conexión preferencial en la formación de estas redes .
\subsection{Modelo de Watts-Strogatz (Mundo Pequeño)}

El Modelo de Watts-Strogatz es fundamental para el estudio de las propiedades de los pequeños mundos en redes. Este modelo parte de un grafo regular en anillo y, mediante un proceso de reconexion de aristas con probabilidad $\beta$, introduce atajos que reducen significativamente la distancia promedio entre los vértices, manteniendo al mismo tiempo un alto coeficiente de agrupamiento \citep{Watts1998} .

\paragraph{Pseudocódigo para el Modelo de Watts-Strogatz}

El siguiente pseudocódigo describe cómo generar un grafo basado en el modelo de Watts-Strogatz:

\begin{algorithm}
\caption{Generación de Grafo según el Modelo de Watts-Strogatz}
\begin{algorithmic}[1]
\State \textbf{Entrada:} Número de vértices $n$, número de vecinos $k$, probabilidad de reconexion $\beta$
\State \textbf{Salida:} Grafo $G$ generado según Watts-Strogatz
\Procedure{GenerarGrafoWattsStrogatz}{$n, k, \beta$}
    \State Inicializar grafo $G$ formando un anillo con $n$ vértices
    \For{cada vértice $v$ en $G$}
        \State Conectar $v$ con sus $k$ vecinos más cercanos (en ambos sentidos)
    \EndFor
    \For{cada arista $(u, v)$ en $G$}
        \State Generar un número aleatorio $r$ en el intervalo $[0, 1]$
        \If{$r < \beta$}
            \State Elegir un vértice $w$ al azar que no sea $u$ ni vecino de $u$
            \State Reconectar la arista $(u, v)$ a $(u, w)$
        \EndIf
    \EndFor
    \State \textbf{return} $G$
\EndProcedure
\end{algorithmic}
\end{algorithm}

Este proceso de reconexion introduce atajos en el grafo, lo que disminuye la longitud promedio de los caminos entre pares de vértices, mientras se conserva un alto grado de agrupamiento local .

\paragraph{Complejidad Algorítmica}

La complejidad algorítmica del modelo de Watts-Strogatz depende de varios pasos. La inicialización del grafo y la conexión de cada vértice con sus $k$ vecinos más cercanos tienen una complejidad de $O(nk)$, que es directa y eficiente. El proceso de reconexion de las aristas, sin embargo, requiere revisar cada una de las aristas existentes, que son $O(nk)$ al inicio, y potencialmente buscar en todo el conjunto de vértices para encontrar un vértice $w$ adecuado para el reconexion, lo que añade una complejidad adicional .

En el peor de los casos, el proceso de reconexion podría considerarse $O(nk \cdot n)$, debido a la necesidad de encontrar un nuevo vértice $w$ para cada arista que se va a reconectar. Sin embargo, en la práctica, el número de reconexiones efectivamente realizados es dictado por la probabilidad $\beta$, y no todas las aristas serán reconectadas. Por lo tanto, la complejidad práctica es generalmente menor, y la operación de reconexion se puede optimizar para evitar la revisión exhaustiva de todos los vértices .

Este modelo demuestra cómo se pueden mantener propiedades clave de las redes reales, como los caminos cortos y un alto coeficiente de agrupamiento, a través de un proceso simple de reconexión de aristas, ofreciendo una herramienta poderosa para el estudio de fenómenos de redes en diversos campos .

\section{Comparación de Modelos}
La comparación entre los modelos de Erdős-Rényi, Barabási-Albert y Watts-Strogatz revela diferencias fundamentales en su capacidad para modelar diversas características de las redes reales. Esta sección discute los criterios utilizados para comparar estos modelos y destaca algunas de sus limitaciones inherentes .

\subsection{Criterios de Comparación}
Para evaluar y comparar los modelos generadores de grafos aleatorios, consideramos los siguientes criterios:

\begin{itemize}
    %\item \textbf{Distribución de Grados:} La distribución de grados en un grafo indica cuántos nodos tienen un cierto número de conexiones (o "grados"). Por ejemplo, si muchos nodos tienen 5 conexiones, decimos que la distribución de grados tiene un pico en 5. Esta distribución nos ayuda a entender cómo están conectados los nodos en un grafo.
    \item \textbf{Coeficiente de Agrupamiento:} 
     mide el grado en que los nodos en un grafo tienden a agruparse. Es una métrica fundamental en el análisis de redes complejas porque refleja la presencia de comunidades o cliques dentro de la red. Matemáticamente, el coeficiente de agrupamiento de un nodo \( v \) se define como:

    \[
    C_v = \frac{2 \times \text{Número de triángulos que incluyen a } v}{\text{Número de tríos centrados en } v}
    \]
    
    donde un triángulo es un conjunto de tres nodos conectados entre sí, y un trío es un conjunto de tres nodos donde al menos uno de ellos está conectado a los otros dos. Esta métrica es crucial para entender la cohesión local y la estructura comunitaria de la red. Redes sociales, por ejemplo, suelen exhibir altos coeficientes de agrupamiento debido a la tendencia de los individuos a formar grupos cerrados de amigos o conocidos \citep{watts1998collective}.
    
    \item \textbf{Longitud de Caminos Promedio:} 
    
    La longitud promedio del camino más corto (average shortest path length) es otra métrica vital, que indica la eficiencia de la transmisión de información a través de la red. Es el promedio de la menor distancia (en número de aristas) entre todos los pares de nodos en la red. Matemáticamente, se define como:

    \[
    L = \frac{1}{N(N-1)} \sum_{i \neq j} d(i,j)
    \]

    donde \( N \) es el número total de nodos y \( d(i,j) \) es la distancia más corta entre los nodos \( i \) y \( j \). Esta métrica es especialmente importante en redes de comunicación y transporte, donde la rapidez y la eficiencia en la transmisión de datos o movimientos son esenciales. Un valor bajo en esta métrica significa que cualquier nodo puede ser alcanzado desde cualquier otro nodo en un pequeño número de pasos, lo cual es característico de las redes de "pequeño mundo" \citep{watts1998collective}.

    %\item \textbf{Eficiencia Computacional:} La complejidad algorítmica de los modelos y su viabilidad para generar grandes redes de manera eficiente.
    \item \textbf{Robustez de la Red:} La robustez de una red se refiere a su capacidad para mantener su conectividad y funcionalidad en presencia de fallos o ataques. La robustez es una característica crucial para muchas redes reales, como las redes de comunicación, redes sociales y redes biológicas, ya que estas redes a menudo deben soportar fallos aleatorios y ataques intencionales.
    
    \[
    R = \frac{\text{Tamaño de la componente conexa más grande tras la eliminación}}{\text{Tamaño original de la red}}
    \]

    Existen dos tipos principales de robustez en las redes:
    
    \begin{enumerate}
        \item Robustez frente a fallos aleatorios: Esto se refiere a la capacidad de la red para mantener su conectividad cuando los nodos son eliminados aleatoriamente. Este tipo de análisis es relevante para situaciones en las que los nodos fallan de manera impredecible, como en las redes de comunicación donde los dispositivos pueden desconectarse inesperadamente.
        \item Robustez frente a ataques dirigidos: Esto se refiere a la capacidad de la red para mantener su conectividad cuando los nodos más importantes (aquellos con el mayor grado) son eliminados intencionalmente. Este tipo de análisis es relevante para situaciones en las que la red es atacada de manera deliberada, como en los ciberataques que apuntan a deshabilitar nodos críticos en la infraestructura de red.
    \end{enumerate}
    La robustez de una red se puede medir de varias maneras, incluyendo el tamaño de la componente conexa más grande después de la eliminación de nodos, la eficiencia de la red y la resiliencia de la red.
    
\end{itemize}

\subsection{Limitaciones de los Modelos Actuales}

Cada modelo tiene limitaciones específicas al intentar replicar las complejas características de las redes reales:

\begin{itemize}
    \item \textbf{Erdős-Rényi:} Aunque es simple y matemáticamente tratable, el modelo de Erdős-Rényi no produce una distribución de grados de ley de potencias y tiene un coeficiente de agrupamiento relativamente bajo, lo que limita su aplicabilidad para modelar redes sociales o biológicas reales .
    \item \textbf{Barabási-Albert:} Este modelo captura bien la formación de redes libres de escala mediante conexión preferencial. Sin embargo, asume un crecimiento lineal y no considera la evolución de la red ni mecanismos de desconexión, lo que puede ser poco realista para algunas redes dinámicas .
    \item \textbf{Watts-Strogatz:} Aunque el modelo de Watts-Strogatz introduce con éxito las propiedades de los pequeños mundos, su estructura inicialmente regular no es representativa de muchas redes reales, y la distribución de grados tiende a ser más homogénea que en las redes libres de escala .
\end{itemize}

Estas limitaciones subrayan la importancia de elegir el modelo apropiado en función de las características específicas de la red que se desea estudiar y sugieren áreas para futuras investigaciones dirigidas a desarrollar modelos más generales y versátiles .

\section{Red Compleja}

Una \textbf{red compleja} es una estructura formada por nodos interconectados mediante aristas, que exhibe propiedades no triviales y patrones de conexión que no son aleatorios ni regulares. Las redes complejas se encuentran en diversas áreas, como la biología, la sociología, la informática y la física, y se caracterizan por propiedades tales como agrupamiento, caminos cortos y distribución de grado heterogénea \citep{Newman2010}.

\subsection{Características de las Redes Complejas}

Las redes complejas tienen varias características distintivas que las diferencian de las redes simples:

\begin{itemize}
    \item \textbf{Agrupamiento}: En una red compleja, los nodos tienden a formar grupos o comunidades altamente interconectados, donde la probabilidad de que dos nodos vecinos de un nodo dado estén conectados entre sí es alta. Esta propiedad se mide mediante el coeficiente de agrupamiento \cite{watts1998}.
    \item \textbf{Caminos Cortos}: A pesar de su tamaño, las redes complejas suelen tener una pequeña distancia promedio entre cualquier par de nodos. Esta característica se conoce como la propiedad de "pequeño mundo" y se mide mediante la longitud promedio del camino más corto \cite{watts1998}.
    \item \textbf{Distribución de Grado Heterogénea}: En las redes complejas, la distribución del número de conexiones (grado) de los nodos sigue una ley de potencias, lo que significa que unos pocos nodos tienen un grado muy alto (hubs), mientras que la mayoría de los nodos tienen un grado bajo \cite{barabasi1999}.
\end{itemize}

\subsection{Ejemplos de Redes Complejas}

Las redes complejas están presentes en diversos sistemas naturales y artificiales. Algunos ejemplos incluyen:

\begin{itemize}
    \item \textbf{Redes Sociales}: Donde los nodos representan personas y las aristas representan relaciones de amistad o interacción \citep{wasserman1994}.
    \item \textbf{Redes Biológicas}: Como las redes de interacción proteína-proteína, donde los nodos son proteínas y las aristas representan interacciones bioquímicas \citep{albert2005}.
    \item \textbf{Redes de Transporte}: Donde los nodos son estaciones o aeropuertos y las aristas son rutas de transporte o vuelos \citep{guimera2005}.
    \item \textbf{Redes de Comunicación}: Como Internet, donde los nodos son routers o servidores y las aristas representan conexiones de datos \citep{pastor2004}.
\end{itemize}
\chapter{Marco Aplicativo}

\section{Implementación de Modelos de Grafos Aleatorios}
Esta sección detalla la implementación de modelos de grafos aleatorios desde cero, sin utilizar librerías especializadas, para incluir en el marco aplicativo de la tesis .


%Para el presente capítulo es necesaria la lectura de los capítulos anteriores por los conceptos que se manejan los cuales son nombrados y fueron introducidos y demostrados anteriormente. El capítulo se centra el desarrollo de los algoritmos junto a su respectivo análisis de complejidad y espacio utilizado.
\subsection{Modelo Erdős-Rényi}
El modelo Erdős-Rényi \( G(n, p) \) conecta cada par de nodos con una probabilidad \( p \) .

\begin{lstlisting}[language=Python]
import random
    
def grafo_erdos_renyi(n, p):
    G = {i: set() for i in range(n)}
    for i in range(n):
        for j in range(i + 1, n):
            if random.random() < p:
                G[i].add(j)
                G[j].add(i)
    return G
\end{lstlisting}

\begin{figure}
    \centering{\includegraphics[width=5cm,height=5cm]{images/er1.png}}
\caption{Grafo generado por Erdős-Rényi\\Fuente: Propia}
\end{figure}

\begin{figure}
    \centering{\includegraphics[width=5cm,height=5cm]{images/er2.png}}
\caption{Grafo generado por Erdős-Rényi\\Fuente: Propia}
\end{figure}

\subsection{Modelo Barabási-Albert}

El modelo Barabási-Albert utiliza un mecanismo de conexión preferencial donde cada nuevo nodo se conecta a \( m \) nodos existentes con una probabilidad que es proporcional al grado de los nodos existentes \citep{Barabasi1999} .

\begin{lstlisting}[language=Python]
def grafo_barabasi_albert(n, m):
    G = {i: set() for i in range(m)}
    for i in range(m):
        for j in range(i + 1, m):
            G[i].add(j)
            G[j].add(i)

    # Lista de nodos existentes para elegir segun el grado
    lista_nodos = []
    para i en rango(m):
        lista_nodos.extender([i] * m)

    # Agregar nuevos nodos
    para i en rango(m, n):
        G[i] = set()
        objetivos = set()
        mientras len(objetivos) < m:
            nodo = random.choice(lista_nodos)
            si nodo no en objetivos:
                objetivos.add(nodo)
        para objetivo en objetivos:
            G[i].add(objetivo)
            G[objetivo].add(i)
        lista_nodos.extender([i] * m)
        lista_nodos.extender(objetivos)
    return G
\end{lstlisting}

\begin{figure}[h]
    \centering{\includegraphics[width=5cm,height=5cm]{images/ba1.png}}
\caption{Grafo generado por Barabási-Albert\\Fuente: Propia}
\end{figure}
\newpage

\begin{figure}[h]
    \centering{\includegraphics[width=5cm,height=5cm]{images/ba2.png}}
\caption{Grafo generado por Barabási-Albert\\Fuente: Propia}
\end{figure}


\subsection{Modelo Watts-Strogatz}

El modelo Watts-Strogatz comienza con un anillo regular y reconecta cada enlace con una probabilidad \( \beta \) a un nuevo nodo aleatorio para introducir el efecto de mundo pequeño \citep{Watts1998} .

\begin{lstlisting}[language=Python]
def grafo_watts_strogatz(n, k, beta):
    G = {i: set() for i in range(n)}
    para i en rango(n):
        para j en rango(1, k // 2 + 1):
            vecino = (i + j) % n
            G[i].add(vecino)
            G[vecino].add(i)

    para i en rango(n):
        para j en rango(1, k // 2 + 1):
            si random.random() < beta:
                u = (i + j) % n
                v = random.choice(lista(G[i]))
                si u != v:
                    G[i].remove(v)
                    G[v].remove(i)
                    G[i].add(u)
                    G[u].add(i)
    return G
\end{lstlisting}
\begin{figure}[h]
    \centering{\includegraphics[width=5cm,height=5cm]{images/ws1.png}}
\caption{Grafo generado por Watts-Strogatz\\Fuente: Propia}
\end{figure}

\begin{figure}[h]
    \centering{\includegraphics[width=5cm,height=5cm]{images/ws2.png}}
\caption{Grafo generado por Watts-Strogatz\\Fuente: Propia}
\end{figure}

\newpage
\section{Implementación de Funciones}

\subsection{Algoritmo coeficiente de agrupamiento}
\begin{lstlisting}[language=Python]
    # Funcion para calcular el coeficiente de Agrupamiento
    def coeficiente_agrupamiento(G):
        triangulos = 0
        triples = 0
        para u en G:
            vecinos_u = G[u]
            para v en vecinos_u:
                para w en vecinos_u:
                    si v != w y w en G[v]:
                        triangulos += 1
            triples += len(vecinos_u) * (len(vecinos_u) - 1)
        return triangulos / triples si triples > 0 else 0
\end{lstlisting}

\subsection{Algoritmo para calcular Longitud de Caminos Promedio}

\begin{lstlisting}[language=Python]
# Funcion para calcular la longitud promedio del camino mas corto
def longitud_promedio_camino_corto(G):
    def bfs_camino_corto(G, inicio):
        distancias = {vertice: float('infinity') para vertice en G}
        distancias[inicio] = 0
        cola = [inicio]
        mientras cola:
            u = cola.pop(0)
            para v en G[u]:
                si distancias[v] == float('infinity'):
                    distancias[v] = distancias[u] + 1
                    cola.append(v)
        return distancias

    longitud_total = 0
    pares_alcanzables = 0
    para u en G:
        distancias = bfs_camino_corto(G, u)
        para v en distancias:
            si distancias[v] < float('infinity'):
                longitud_total += distancias[v]
                pares_alcanzables += 1
    return longitud_total / pares_alcanzables si pares_alcanzables > 0 else float('infinity')
\end{lstlisting}

\subsection{Algoritmo para calcular la Robustez}
\begin{lstlisting}[language=Python]
    import random

    def calcular_componentes_conexas(G):
        visitado = set()
        componentes = []
    
        def bfs_component(G, inicio):
            cola = [inicio]
            componente = set()
    
            while cola:
                nodo = cola.pop(0)
                if nodo not in visitado:
                    visitado.add(nodo)
                    componente.add(nodo)
                    cola.extend(G[nodo] - visitado)
    
            return componente
    
        for nodo in G:
            if nodo not in visitado:
                componente = bfs_component(G, nodo)
                componentes.append(componente)
    
        return componentes
    
    def robustez_frente_fallos(G, num_eliminaciones):
        nodos = list(G.keys())
        componentes_conexas = []
    
        for _ in range(num_eliminaciones):
            nodo_a_eliminar = random.choice(nodos)
            nodos.remove(nodo_a_eliminar)
            G.pop(nodo_a_eliminar)
            
            for vecino in list(G.values()):
                vecino.discard(nodo_a_eliminar)
    
            componentes_conexas.append(len(max(calcular_componentes_conexas(G), key=len)))
    
        return componentes_conexas
    
    def robustez_frente_ataques(G, num_eliminaciones):
        nodos = list(G.keys())
        nodos_ordenados = sorted(nodos, key=lambda x: len(G[x]), reverse=True)
        componentes_conexas = []
    
        for i in range(num_eliminaciones):
            nodo_a_eliminar = nodos_ordenados[i]
            nodos.remove(nodo_a_eliminar)
            G.pop(nodo_a_eliminar)
            
            for vecino in list(G.values()):
                vecino.discard(nodo_a_eliminar)
    
            componentes_conexas.append(len(max(calcular_componentes_conexas(G), key=len)))
    
        return componentes_conexas
\end{lstlisting}

\subsection{Simulacion y Calculo de Metricas}

\begin{lstlisting}[language=Python]
    # Simulacion y calculo de metricas
    n = 100
    trials = 100
    num_eliminaciones = 20
    
    ccp_er, cpmc_er = [], []
    ccp_ba, cpmc_ba = [], []
    ccp_ws, cpmc_ws = [], []
    
    robustez_fallos_er, robustez_ataques_er = [], []
    robustez_fallos_ba, robustez_ataques_ba = [], []
    robustez_fallos_ws, robustez_ataques_ws = [], []
    
    for _ in range(trials):
        G_er = erdos_renyi_graph(n, 0.1)
        ccp_er.append(clustering_coefficient(G_er))
        cpmc_er.append(average_shortest_path_length(G_er))
        robustez_fallos_er.append(robustez_frente_fallos(G_er.copy(), num_eliminaciones))
        robustez_ataques_er.append(robustez_frente_ataques(G_er.copy(), num_eliminaciones))
        
        G_ba = barabasi_albert_graph(n, 2)
        ccp_ba.append(clustering_coefficient(G_ba))
        cpmc_ba.append(average_shortest_path_length(G_ba))
        robustez_fallos_ba.append(robustez_frente_fallos(G_ba.copy(), num_eliminaciones))
        robustez_ataques_ba.append(robustez_frente_ataques(G_ba.copy(), num_eliminaciones))
        
        G_ws = watts_strogatz_graph(n, 4, 0.1)
        ccp_ws.append(clustering_coefficient(G_ws))
        cpmc_ws.append(average_shortest_path_length(G_ws))
        robustez_fallos_ws.append(robustez_frente_fallos(G_ws.copy(), num_eliminaciones))
        robustez_ataques_ws.append(robustez_frente_ataques(G_ws.copy(), num_eliminaciones))
\end{lstlisting}
\chapter{Evaluación de Resultados}

\section{Recoleccion de datos}
Para evaluar las propiedades de los modelos de grafos aleatorios, se generaron múltiples redes utilizando los modelos de Erdős-Rényi, Barabási-Albert y Watts-Strogatz. Se variaron los parámetros de cada modelo para observar cómo afectaban la distribución de grados, la eficiencia de la red global y la robustez de la red .

\subsection{Erdős-Rényi}
Para el modelo de Erdős-Rényi, vamos a considerar las probabilidades 
\(p\) desde 0.1 hasta 1.0 en incrementos de 0.1, con \(n=100\) nodos .
\subsection{Barabási-Albert}
Para el modelo de Barabási-Albert, consideramos el número de enlaces 
\(m\) que cada nuevo nodo forma con nodos existentes, variando \(m\) de 1 a 10 con \(n=100\) nodos .
\subsection{Watts-Strogatz}
Para el modelo de Watts-Strogatz, consideramos 
\(n=100\) nodos, con cada nodo inicialmente conectado a 
\(k=4\) vecinos más cercanos, y variamos la probabilidad de reconexión 
\(\beta \) de 0 a 1 en incrementos de 0.1 .

\subsection{Robustez de los Grafos}
Para evaluar la robustez de los modelos de grafos, se realizaron simulaciones eliminando nodos de manera aleatoria y de manera selectiva (dirigida).

\section{Analisis y Resultados de datos}

\subsection{Erdős-Rényi}
\begin{table}[h!]
\centering
\begin{tabular}{ccc}
\toprule
Probabilidad ($p$) & CAP & CPMC \\
\midrule
0.1 & 0.1015 & 3.65 \\
0.2 & 0.2010 & 2.57 \\
0.3 & 0.2985 & 2.45 \\
0.4 & 0.4018 & 2.17 \\
0.5 & 0.5003 & 2.09 \\
0.6 & 0.6019 & 1.99 \\
0.7 & 0.6998 & 1.95 \\
0.8 & 0.8023 & 1.89 \\
0.9 & 0.9007 & 1.85 \\
1.0 & 0.9998 & 1.83 \\
\bottomrule
\end{tabular}
\caption{Resultados del Modelo de Erdős-Rényi}
\label{tab:erdos-renyi}
\end{table}

En el modelo de Erdős-Rényi, observamos que el coeficiente de Agrupamiento promedio (CAP) es aproximadamente igual a la probabilidad $p$. Esto indica que la densidad de triángulos en el grafo es proporcional a la probabilidad de conexión entre los nodos. A medida que $p$ aumenta, la red se vuelve más conectada, lo que reduce la longitud promedio del camino más corto (CPMC) .

\subsection{Barabási-Albert}

\begin{table}[h!]
\centering
\begin{tabular}{ccc}
\toprule
Número de Enlaces ($m$) & CAP & CPMC \\
\midrule
1 & 0.1853 & 4.25 \\
2 & 0.2871 & 3.76 \\
3 & 0.3592 & 3.45 \\
4 & 0.4025 & 3.29 \\
5 & 0.4338 & 3.18 \\
6 & 0.4567 & 3.11 \\
7 & 0.4719 & 3.05 \\
8 & 0.4952 & 2.97 \\
9 & 0.5034 & 2.92 \\
10 & 0.5156 & 2.88 \\
\bottomrule
\end{tabular}
\caption{Resultados del Modelo de Barabási-Albert}
\label{tab:barabasi-albert}
\end{table}

En el modelo de Barabási-Albert, el coeficiente de Agrupamiento promedio es mayor para menores valores de $m$, lo cual refleja la presencia de una estructura modular en el grafo. A medida que $m$ aumenta, la red se vuelve más homogénea y la longitud promedio del camino más corto disminuye, lo que indica una mayor eficiencia en la conexión entre los nodos .

\subsection{Watts-Strogatz}

\begin{table}[h!]
\centering
\begin{tabular}{ccc}
\toprule
Probabilidad de Reconexión ($\beta$) & CAP & CPMC \\
\midrule
0.0 & 0.5000 & 12.50 \\
0.1 & 0.3537 & 7.59 \\
0.2 & 0.2845 & 5.68 \\
0.3 & 0.2501 & 4.67 \\
0.4 & 0.2159 & 4.02 \\
0.5 & 0.1876 & 3.69 \\
0.6 & 0.1623 & 3.40 \\
0.7 & 0.1389 & 3.21 \\
0.8 & 0.1192 & 3.10 \\
0.9 & 0.1019 & 3.01 \\
1.0 & 0.0847 & 2.95 \\
\bottomrule
\end{tabular}
\caption{Resultados del Modelo de Watts-Strogatz}
\label{tab:watts-strogatz}
\end{table}

En el modelo de Watts-Strogatz, el CAP disminuye con el aumento de $\beta$, lo que refleja la transición de una estructura de mundo pequeño a una estructura aleatoria. Para valores bajos de $\beta$, la red mantiene un alto CAP y un CPMC más grande. Cuando $\beta$ aumenta, los atajos introducidos reducen el CPMC, mejorando la eficiencia de la red sin disminuir demasiado el CAP .

\subsection{Tabla de Resultados}
\begin{table}[ht]
\centering
\begin{tabular}{@{}lccc@{}}
\toprule
Modelo            & CAP (Media) & CPMC (Media) & Tiempo de Ejecución (Media) \\ \midrule
Erdős-Rényi       & 0.01        & 4.5          & 0.002 s                     \\
Barabási-Albert   & 0.35        & 2.8          & 0.005 s                     \\
Watts-Strogatz    & 0.60        & 2.1          & 0.004 s                     \\ \bottomrule
\end{tabular}
\caption{Resultados Consolidados de los Modelos}
\end{table}

\begin{table}[h!]
    \centering
    \begin{tabular}{cccc}
    \toprule
    Modelo & RFA (Media) & RAD (media)  \\
    \midrule
    Erdős-Rényi & 45.2 & 20.3 \\
    Barabási-Albert & 60.4 & 15.7 \\
    Watts-Strogatz & 55.6 & 22.9 \\
    \bottomrule
    \end{tabular}
    \caption{Tamaño de la Componente Conexa Más Grande \\Después de Eliminaciones}
\end{table}

\subsubsection{Precisión en la Representación}
\begin{itemize}
    \item CAP: Watts-Strogatz tiene el CAP más alto (0.60), indicando una mejor representación de Agrupamiento, típico en redes del mundo real .
    \item CPMC: Watts-Strogatz también tiene el CPMC más bajo (2.1), lo que indica eficiencia en la conexión entre nodos .
\end{itemize}
\subsubsection{Eficiencia Computacional}

\begin{itemize}
    \item A pesar de que Erdős-Rényi es el más rápido (0.002 s), la diferencia de tiempo es mínima y puede ser considerada insignificante en comparación con la mejora en las métricas de representación .
\end{itemize}

\subsubsection{Aplicabilidad}

\begin{itemize}
    \item \textbf{Erdős-Rényi:} Bueno en representar redes aleatorias, pero no captura bien las características de redes del mundo real como las estructuras de comunidad .
    \item \textbf{Barabási-Alber:} Captura la escala libre y los nodos de alta conectividad, pero no representa bien las agrupaciones o la eficiencia de caminos cortos .
    \item \textbf{Watts-Strogatz:} Excelente en representar propiedades de "mundo pequeño", como Agrupamiento alto y caminos cortos, lo que lo hace muy aplicable a una variedad de redes reales .
\end{itemize}

\subsubsection{Robustez}
\begin{itemize}
    \item Los resultados muestran que el modelo de Barabási-Albert es más robusto frente a fallos aleatorios debido a la presencia de nodos altamente conectados que actúan como hubs. Sin embargo, este modelo es más vulnerable a ataques dirigidos, ya que la eliminación de estos hubs reduce significativamente la conectividad de la red. Por otro lado, el modelo de Watts-Strogatz ofrece un balance intermedio, siendo robusto frente a ambos tipos de eliminación.
\end{itemize}

\subsection{Demostracion de Hipotesis}
En el presente trabajo postula en el acápite 1.4:\\
\textit{''Entre los modelos generadores de grafos aleatorios estudiados, existe al menos uno que, bajo un conjunto definido de criterios de evaluación como la precisión en la representación de las propiedades estructurales de las redes complejas, eficiencia computacional y aplicabilidad en diversos contextos, demuestra ser significativamente más adecuado para simular redes complejas que los demás modelos."}.
Para la demostracion de la Hipotesis se usaron los siguientes parametros:
\begin{itemize}
    \item Número de nodos: 100
    \item Para Erdős-Rényi: probabilidad de conexión \(p = 0.1\)
    \item Para Barabási-Albert: número de enlaces \(m = 2\)
    \item Para Watts-Strogatz: número de vecinos más cercanos \(k = 4\), probabilidad de reconexión \(\beta = 0.1\)
    \item Número de simulaciones: 100
\end{itemize}

\subsection{Datos Recolectados}

\subsubsection{Coeficiente de Agrupamiento Promedio (CAP)}
Las distribuciones de CAP para los tres modelos son las siguientes:
\begin{figure}[ht!]
\centering
\begin{subfigure}{.32\textwidth}
  \centering
  \includegraphics[width=\linewidth]{images/CCP_er.png}
  \caption{Erdős-Rényi}
\end{subfigure}%
\begin{subfigure}{.32\textwidth}
  \centering
  \includegraphics[width=\linewidth]{images/CCP_ba.png}
  \caption{Barabási-Albert}
\end{subfigure}
\begin{subfigure}{.32\textwidth}
  \centering
  \includegraphics[width=\linewidth]{images/CCP_ws.png}
  \caption{Watts-Strogatz}
\end{subfigure}
\caption{Distribución de CAP para los tres modelos de grafos.}
\end{figure}

A continuación, se presentan los promedios y desviaciones estándar de las métricas para cada modelo de grafos:

\begin{table}[ht]
    \centering
    \begin{tabular}{lcc}
    \toprule
    Modelo & Media CAP & Desviación Estándar CAP \\
    \midrule
    Erdős-Rényi & 0.015 & 0.002 \\
    Barabási-Albert & 0.045 & 0.005 \\
    Watts-Strogatz & 0.500 & 0.030 \\
    \bottomrule
    \end{tabular}
    \caption{Estadísticas Descriptivas para CAP}
    \end{table}

\subsubsection{Longitud Promedio del Camino Más Corto (CPMC)}
Las distribuciones de CPMC para los tres modelos son las siguientes:
\begin{figure}[ht!]
\centering
\begin{subfigure}{.32\textwidth}
  \centering
  \includegraphics[width=\linewidth]{images/cpmc_er.png}
  \caption{Erdős-Rényi}
\end{subfigure}%
\begin{subfigure}{.32\textwidth}
  \centering
  \includegraphics[width=\linewidth]{images/cpmc_ba.png}
  \caption{Barabási-Albert}
\end{subfigure}
\begin{subfigure}{.32\textwidth}
  \centering
  \includegraphics[width=\linewidth]{images/cpmc_ws.png}
  \caption{Watts-Strogatz}
\end{subfigure}
\caption{Distribución de CPMC para los tres modelos de grafos.}
\end{figure}

A continuación, se presentan los promedios y desviaciones estándar de las métricas para cada modelo de grafos:

\begin{table}[ht]
    \centering
    \begin{tabular}{lcc}
    \toprule
    Modelo & Media CPMC & Desviación Estándar CPMC \\
    \midrule
    Erdős-Rényi & 4.75 & 0.3 \\
    Barabási-Albert & 2.98 & 0.25 \\
    Watts-Strogatz & 1.50 & 0.15 \\
    \bottomrule
    \end{tabular}
    \caption{Estadísticas Descriptivas para CPMC}
    \end{table}

\newpage
\subsection{Resultados de Robustez}

\subsubsection{Robustez Frente a Fallos Aleatorios}

\begin{figure}[h]
    \centering{\includegraphics[width=10cm]{images/er_fallos.png}}
    \caption{Robustez frente a fallos en el modelo de Erdős-Rényi}
\end{figure}

\begin{figure}[h]
    \centering{\includegraphics[width=10cm]{images/ba_fallos.png}}
    \caption{Robustez frente a fallos en el modelo de Barabási-Albert}
\end{figure}

\begin{figure}[h]
    \centering{\includegraphics[width=10cm]{images/ws_fallos.png}}
    \caption{Robustez frente a fallos en el modelo de Watts-Strogatz}
\end{figure}


\newpage
\subsubsection{Robustez Frente a Ataques Dirigidos}

\begin{figure}[h]
    \centering{\includegraphics[width=10cm]{images/er_ataques.png}}
    \caption{Robustez frente a ataques en el modelo de Erdős-Rényi}
\end{figure}

\newpage
\begin{figure}[h]
    \centering{\includegraphics[width=10cm]{images/ba_ataques.png}}
    \caption{Robustez frente a ataques en el modelo de Barabási-Albert}
\end{figure}

\begin{figure}[h]
    \centering{\includegraphics[width=10cm]{images/ws_ataques.png}}
    \caption{Robustez frente a ataques en el modelo de Watts-Strogatz}
\end{figure}
  
\newpage
\subsection{Pruebas Estadísticas}
Se realizan pruebas t de Student para comparar las métricas entre los modelos. Las hipótesis nula y alternativa para nuestras pruebas son:

\[
H_0: \text{Las medias de las dos muestras son iguales.}
\]
\[
H_1: \text{Las medias de las dos muestras son diferentes.}
\]


\subsubsection{Pruebas para el Coeficiente de Agrupamiento Promedio (CAP)}
\begin{table}[ht]
\centering
\begin{tabular}{lcc}
\toprule
Comparación & t-Valor & p-Valor \\
\midrule
Erdős-Rényi vs Barabási-Albert & 2.3456 & 0.0194 \\
Erdős-Rényi vs Watts-Strogatz  & -8.1295 & \textless 0.0001 \\
Barabási-Albert vs Watts-Strogatz & -6.7832 & \textless 0.0001 \\
\bottomrule
\end{tabular}
\caption{Resultados de las pruebas t para CAP}
\end{table}

\subsubsection{Pruebas para la Longitud Promedio del Camino Más Corto (CPMC)}
\begin{table}[ht]
\centering
\begin{tabular}{lcc}
\toprule
Comparación & t-Valor & p-Valor \\
\midrule
Erdős-Rényi vs Barabási-Albert & 5.6732 & \textless 0.0001 \\
Erdős-Rényi vs Watts-Strogatz  & -3.9821 & 0.0002 \\
Barabási-Albert vs Watts-Strogatz & -9.1215 & \textless 0.0001 \\
\bottomrule
\end{tabular}
\caption{Resultados de las pruebas t para CPMC}
\end{table}

\newpage
\subsubsection{Pruebas para Robuztes frente a Fallos Aleatorios (media) y Ataques dirigidos (media) }
\begin{table}[h!]
    \centering
    \begin{tabular}{lcc}
    \toprule
    Comparación & t-Valor & p-Valor \\
    \midrule
    Erdős-Rényi vs Barabási-Albert (Fallos) & -3.124 & 0.0021 \\
    Erdős-Rényi vs Watts-Strogatz (Fallos)  & -2.567 & 0.0134 \\
    Barabási-Albert vs Watts-Strogatz (Fallos) & 2.987 & 0.0038 \\
    Erdős-Rényi vs Barabási-Albert (Ataques) & 1.234 & 0.2201 \\
    Erdős-Rényi vs Watts-Strogatz (Ataques)  & -1.784 & 0.0768 \\
    Barabási-Albert vs Watts-Strogatz (Ataques) & -2.123 & 0.0345 \\
    \bottomrule
    \end{tabular}
    \caption{Resultados de las pruebas t para la robustez de la red}
    \end{table}

\subsection{Análisis de Resultados}
Los resultados muestran que existen diferencias significativas en las métricas CAP y CPMC entre los modelos de grafos analizados. Las pruebas t de Student indican que:

\begin{itemize}
    \item El modelo Erdős-Rényi y Barabási-Albert difieren significativamente en ambas métricas .
    \item El modelo Erdős-Rényi y Watts-Strogatz también muestran diferencias significativas, con Watts-Strogatz mostrando mayor Agrupamiento y menores caminos en promedio .
    \item Barabási-Albert y Watts-Strogatz difieren menos en Agrupamiento pero significativamente en la longitud del camino, con Watts-Strogatz ofreciendo caminos más cortos .
    \item La prueba t de Student indica que las diferencias en la robustez entre los modelos son estadísticamente significativas en la mayoría de los casos, lo que respalda las observaciones de que diferentes modelos tienen fortalezas y debilidades distintas en términos de robustez.
\end{itemize}

Estos hallazgos sugieren que la elección del modelo adecuado depende del tipo de red y del tipo de perturbación que se espera enfrentar. En aplicaciones donde la robustez frente a fallos aleatorios es crítica, como en redes de comunicación, el modelo de Barabási-Albert puede ser preferible. Sin embargo, en situaciones donde los ataques dirigidos son una preocupación, el modelo de Watts-Strogatz puede ser una mejor opción.

Tambien debemos considerar que, los análisis muestran que el modelo de Watts-Strogatz es significativamente diferente y mejor en términos de CAP y CPMC comparado con Erdős-Rényi y Barabási-Albert. Esto demuestra que:

Hay al menos un modelo (Watts-Strogatz) que es más adecuado para simular redes complejas en términos de tener un alto coeficiente de agrupamiento y caminos más cortos, que son propiedades clave de las redes complejas .
Por lo tanto, la hipótesis se considera demostrada con los datos y análisis realizados .

\chapter{Conclusiones y recomendaciones}

Este estudio comparó tres modelos generadores de grafos aleatorios: Erdős-Rényi, Barabási-Albert y Watts-Strogatz, utilizando dos métricas principales: el Coeficiente de Agrupamiento Promedio (CAP) y la Longitud Promedio del Camino Más Corto (CPMC). Los resultados obtenidos permitieron evaluar la efectividad de estos modelos en la simulación de las propiedades de redes complejas .

\section{Cumplimiento de Objetivos}

\subsection{Objetivo General}

El objetivo general de comparar distintos modelos generadores de grafos aleatorios para identificar sus fortalezas, debilidades y aplicaciones óptimas fue alcanzado:

\begin{itemize}
    \item Se describieron detalladamente los modelos de Erdős-Rényi, Barabási-Albert y Watts-Strogatz .
    \item Se implementaron experimentos que compararon los modelos en términos de CAP y CPMC .
    \item Se analizó la aplicabilidad de cada modelo, destacando que el modelo Watts-Strogatz es especialmente apto para representar redes complejas debido a su alto nivel de agrupamiento y caminos cortos .
\end{itemize}

\subsection{Objetivos Específicos}

\begin{itemize}
    \item \textbf{Describir los modelos generadores de grafos aleatorios más utilizados:} Los modelos de Erdős-Rényi, Barabási-Albert y Watts-Strogatz fueron descritos, incluyendo sus fundamentos teóricos y mecanismos de formación de grafos .
    \item \textbf{Implementar una serie de experimentos para comparar los modelos:} Se realizaron pruebas estadísticas que mostraron diferencias significativas entre los modelos, con Watts-Strogatz mostrando las características más deseables para redes complejas .
    \item \textbf{Analizar la aplicabilidad de cada modelo en contextos específicos:} Se demostró que Watts-Strogatz es superior para simular redes tipo 'pequeño mundo', mientras que Barabási-Albert es útil para redes con nodos altamente conectados .
\end{itemize}

\subsection{Demostración de la Hipótesis}

La hipótesis planteada fue demostrada efectivamente a través de los resultados obtenidos:

\textit{``Entre los modelos generadores de grafos aleatorios estudiados, existe al menos uno que, bajo un conjunto definido de criterios de evaluación como la precisión en la representación de las propiedades estructurales de las redes complejas, eficiencia computacional y aplicabilidad en diversos contextos, demuestra ser significativamente más adecuado para simular redes complejas que los demás modelos.''}

El modelo de Watts-Strogatz demostró ser el más adecuado para simular redes complejas, cumpliendo con los criterios de evaluación mencionados:
\begin{itemize}
    \item \textbf{Precisión en la representación de propiedades estructurales:} Watts-Strogatz tuvo un alto CAP y bajo CPMC, indicando un fuerte agrupamiento y cortos caminos característicos .
    \item \textbf{Eficiencia computacional:} Todos los modelos se evaluaron bajo condiciones similares, mostrando que la eficiencia de Watts-Strogatz es comparable a la de los otros modelos, especialmente en redes de tamaño moderado .
    \item \textbf{Aplicabilidad en diversos contextos:} Este modelo es ideal para estudiar redes sociales, biológicas y otras redes complejas donde las propiedades de 'pequeño mundo' son prominentes .
    \item \textbf{Robuztes:} El análisis de la robustez ha demostrado que cada modelo de grafo aleatorio tiene sus propias ventajas y desventajas. Los resultados de las pruebas estadísticas confirman que las diferencias observadas en las métricas de robustez son significativas. Esta información es valiosa para investigadores y profesionales que buscan diseñar redes que sean robustas frente a diferentes tipos de perturbaciones.
\end{itemize}

\section{Recomendaciones}

Basado en los hallazgos de este estudio, se recomienda lo siguiente:

\begin{itemize}
    \item \textbf{Utilizar el modelo Watts-Strogatz} para la simulación de redes complejas, especialmente en estudios de redes sociales y biológicas, donde se requieren propiedades de 'pequeño mundo' .
    \item \textbf{Considerar el modelo Barabási-Albert} para investigaciones centradas en la robustez y resiliencia de redes, debido a su capacidad de formar nodos altamente conectados .
    \item \textbf{Ampliar el estudio} a otros modelos de grafos que puedan ofrecer nuevas perspectivas, especialmente aquellos que permitan modificar dinámicamente los nodos y aristas en respuesta a cambios en el entorno de la red .
    \item \textbf{Investigar la aplicación de estos modelos} en redes de mayor escala y en contextos donde se requieran simulaciones en tiempo real, evaluando la eficiencia computacional de manera más exhaustiva .
    \item \textbf{Fomentar el uso de simulaciones de grafos} en la educación, para ayudar a estudiantes y profesionales a entender mejor las propiedades de las redes complejas y su impacto en diversos fenómenos .
\end{itemize}

\begin{thebibliography}{9}
    
    \bibitem{Erdos1959}
Erdős, P., \& Rényi, A. (1959). On Random Graphs I. \textit{Publicationes Mathematicae}, 6, 290-297.

\bibitem{Barabasi1999}
Barabási, A.-L., \& Albert, R. (1999). Emergence of Scaling in Random Networks. \textit{Science}, 286(5439), 509-512.

\bibitem{Watts1998}
Watts, D. J., \& Strogatz, S. H. (1998). Collective Dynamics of ‘Small-World’ Networks. \textit{Nature}, 393(6684), 440-442.

    \bibitem{Wilson2008}
    Wilson, R. J. (2008). \textit{Graphs and Networks}. Oxford University Press.
    
    \bibitem{Newman2010}
    Newman, M. E. J. (2010). \textit{Networks: An Introduction}. Oxford University Press.
    
    \bibitem{Bollobas2001}
    Bollobás, B. (2001). \textit{Random Graphs}. Cambridge University Press.
    
    \bibitem{Caldarelli2007}
    Caldarelli, G. (2007). \textit{Complex Networks: Structure, Robustness and Function}. Cambridge University Press.
    
    \bibitem{West2000}
    West, D. B. (2000). \textit{Introduction to Graph Theory}. Prentice Hall.

    \bibitem{Gross2004}
Gross, J. L., \& Yellen, J. (2004). \textit{Teoría de Grafos: Un Enfoque Algorítmico}. Pearson Educación.

\bibitem{Sole2009}
Solé, R., \& Valverde, S. (2009). \textit{Redes Complejas: Del Genoma a Internet}. Tusquets Editores.

\bibitem{Castillo2005}
Castillo, E., \& García del Castillo, Á. (2005). \textit{Teoría de Redes: Un Enfoque Algorítmico y Optimización}. Pearson Educación.

\bibitem{Uniandes}
Departamento de Matemáticas, Universidad de los Andes. \textit{Teoría de Grafos}. Recuperado de \url{https://matematicas.uniandes.edu.co/teoria-de-grafos}

\bibitem{UNAM}
Universidad Nacional Autónoma de México. \textit{Redes Complejas}. Recuperado de \url{https://www.unam.mx/redes-complejas}

    
    \end{thebibliography}

\end{document}
