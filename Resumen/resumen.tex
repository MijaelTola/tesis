\doublespacing
\chapter*{\centering \normalsize Resumen}
%\addcontentsline{toc}{chapter}{Resumen}
En esta tesis se realiza una evaluación comparativa de tres modelos generadores de grafos aleatorios: Erdős-Rényi, Barabási-Albert y Watts-Strogatz. 
Estos modelos se analizan en términos de sus propiedades estructurales, como el coeficiente de agrupamiento, la longitud promedio del camino más corto y la robustez frente 
a fallos aleatorios y ataques dirigidos. El modelo de Erdős-Rényi, aunque simple y eficiente, no captura bien las características de agrupamiento observadas en muchas redes reales. 
Por otro lado, el modelo de Barabási-Albert genera redes con una distribución de grado de ley de potencias, lo que es adecuado para representar redes con nodos altamente 
conectados, pero puede no ser ideal para redes con alta cohesión interna. Finalmente, el modelo de Watts-Strogatz, que combina propiedades de redes regulares y aleatorias, 
se destaca por su capacidad para representar redes de "mundo pequeño" con alto agrupamiento y caminos cortos. Los resultados de los experimentos muestran que el modelo 
Watts-Strogatz es el más adecuado para representar redes complejas debido a su alto coeficiente de agrupamiento y caminos cortos. El estudio concluye que cada modelo tiene 
sus fortalezas y debilidades, siendo el modelo de Watts-Strogatz el más robusto frente a perturbaciones, lo que lo hace ideal para aplicaciones en redes sociales, biológicas y de comunicación.\\
\textbf{Palabras clave:} Grafos aleatorios, redes complejas, Erdős-Rényi, Barabási-Albert, Watts-Strogatz, coeficiente de agrupamiento, camino más corto, robustez.


\chapter*{\centering \normalsize Abstract}
%\addcontentsline{toc}{chapter}{Abstract}
This thesis presents a comparative evaluation of three random graph generator models: Erdős-Rényi, Barabási-Albert, and Watts-Strogatz. 
These models are analyzed in terms of their structural properties, such as clustering coefficient, average shortest path length, and robustness against 
random failures and targeted attacks. The Erdős-Rényi model, though simple and efficient, does not well capture the clustering characteristics observed in many 
real networks. Conversely, the Barabási-Albert model generates networks with a power-law degree distribution, suitable for representing networks with highly 
connected hubs but may not be ideal for networks with high internal cohesion. The Watts-Strogatz model, which combines properties of regular and random networks, 
excels in representing "small-world" networks with high clustering and short paths. Experimental results show that the Watts-Strogatz model is the most suitable for representing 
complex networks due to its high clustering coefficient and short paths. The study concludes that each model has its strengths and weaknesses, with the Watts-Strogatz model being
 the most robust against perturbations, making it ideal for applications in social, biological, and communication networks.\\
\textbf{Keywords:} Random graphs, complex networks, Erdős-Rényi, Barabási-Albert, Watts-Strogatz, clustering coefficient, shortest path, robustness.

