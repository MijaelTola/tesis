\begin{thebibliography}{9}
    \bibitem{cormen2009}
Cormen, T. H., Leiserson, C. E., Rivest, R. L., \& Stein, C. (2009). \textit{Introduction to Algorithms} (3rd ed.). MIT Press.

    
    \bibitem{Erdos1959}
Erdős, P., \& Rényi, A. (1959). On Random Graphs I. \textit{Publicationes Mathematicae}, 6, 290-297.

\bibitem{Barabasi1999}
Barabási, A.-L., \& Albert, R. (1999). Emergence of Scaling in Random Networks. \textit{Science}, 286(5439), 509-512.

\bibitem{Watts1998}
Watts, D. J., \& Strogatz, S. H. (1998). Collective Dynamics of ‘Small-World’ Networks. \textit{Nature}, 393(6684), 440-442.

    \bibitem{Wilson2008}
    Wilson, R. J. (2008). \textit{Graphs and Networks}. Oxford University Press.
    
    \bibitem{Newman2010}
    Newman, M. E. J. (2010). \textit{Networks: An Introduction}. Oxford University Press.
    
    \bibitem{Bollobas2001}
    Bollobás, B. (2001). \textit{Random Graphs}. Cambridge University Press.
    
    \bibitem{Caldarelli2007}
    Caldarelli, G. (2007). \textit{Complex Networks: Structure, Robustness and Function}. Cambridge University Press.
    
    \bibitem{West2000}
    West, D. B. (2000). \textit{Introduction to Graph Theory}. Prentice Hall.

    \bibitem{Gross2004}
Gross, J. L., \& Yellen, J. (2004). \textit{Teoría de Grafos: Un Enfoque Algorítmico}. Pearson Educación.

\bibitem{Sole2009}
Solé, R., \& Valverde, S. (2009). \textit{Redes Complejas: Del Genoma a Internet}. Tusquets Editores.

\bibitem{Castillo2005}
Castillo, E., \& García del Castillo, Á. (2005). \textit{Teoría de Redes: Un Enfoque Algorítmico y Optimización}. Pearson Educación.

\bibitem{Uniandes}
Departamento de Matemáticas, Universidad de los Andes. \textit{Teoría de Grafos}. Recuperado de \url{https://matematicas.uniandes.edu.co/teoria-de-grafos}

\bibitem{UNAM}
Universidad Nacional Autónoma de México. \textit{Redes Complejas}. Recuperado de \url{https://www.unam.mx/redes-complejas}

    
    \end{thebibliography}